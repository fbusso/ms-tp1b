\documentclass{article}
\usepackage[spanish]{babel}
\usepackage{graphicx}
\usepackage{mathtools}
\usepackage{vmargin}
\usepackage{pgfplots}
\usepackage{fancyhdr}

\pgfplotsset{compat=newest}
\setpapersize{A4}
\newcommand*\rfrac[2]{{}^{#1}\!/_{#2}}		

\title{Matemática Superior\\Trabajo Práctico 1}

\author{ Busso, Francisco\\francisco.busso@outlook.com 
    \and Ferraro Trivelli, Giovani\\gftrivelli@frsf.utn.edu.ar
    \and Storani, Miguel\\ miguelignaciostorani@gmail.com
}

\pagestyle{fancy}

\renewcommand{\maketitle}{
    \begin{center}
        
        {\scshape{Universidad Tecnológica Nacional\\Facultad Regional Santa Fe}}
        \vspace{10pt}
        \hrule
        \vspace{10pt}
       

        {\LARGE\bfseries{Matemática Superior\\}}
        \vspace{5pt}
        {\Huge\bfseries{Trabajo Práctico 1B}}

        \vspace{8pt}
        \hrule
        \vspace{8pt}

        Busso, Francisco\\
        \textit{francisco.busso@outlook.com}\\
        \vspace{7pt}
        Ferraro Trivelli, Giovani\\
        \textit{gftrivelli@frsf.utn.edu.ar}\\
        \vspace{7pt}
        Storani, Miguel\\
        \textit{miguelignaciostorani@gmail.com}\\

        % Setea el estilo de la pagina a vacío
        \thispagestyle{empty}
        
        % Terimna la pagina actual
        \newpage
    \end{center}
}

\rhead{Matemática Superior}
\lhead{\includegraphics[width=3.2cm]{logo.png}}

\begin{document}
\maketitle
\newpage

\renewcommand{\contentsname}{Índice}
\tableofcontents
\newpage

    \section{Transformada Discerta}
        \paragraph{}
        Una vez separada la función en segmentos y calculadas las Transformadas de Fourier en Tiempo Discreto de cada uno, 
        se construyó la Matriz de Espectro E utilizando la formula propuesta. Posteriormente se determinaron los valores máximo y mínimo de dicha matriz 
        y se construyó una nueva matriz de igual dimensión a la matriz de espectro, en la cuál se le asignó a cada posición de la misma un valor entre 0 y 255, 
        mediante la siguiente fórmula:

        \begin{equation}
            V_{(i, j)}=\frac{E_{(i, j)}-m}{M-m}*255
        \end{equation}

        Donde \textit{M} representa al valor máximo de la Matriz de Espectro y \textit{m} al valor mínimo de la misma.
        Esto se realizó para diferenciar gráficamente las frecuencias predominantes (zonas  blancas) 
        de las frecuencias que aportan menos energía (zonas negras) a la señal final.

        \begin{figure}[h!]
            \centering
            \includegraphics[width=40mm]{color_img}
            \caption{Representación gráfica de la Matriz de Espectro}
            \label{imagen1}
        \end{figure}

        \paragraph{}
        Lo mismo se realizó para una señal simple de frecuencia constante y para una señal con variación de frecuencia lineal.

        \begin{figure}[h!]
            \centering
            \includegraphics[width=40mm]{constante}
            \caption{Representacion gráfica de una señal de frecuencia constante}
            \label{imagen2}
        \end{figure}

        \begin{figure}[h!]
            \centering
            \includegraphics[width=40mm]{variable}
            \caption{Representación gráfica de una señal con variación de frecuencia lineal}
            \label{imagen3}
        \end{figure}

        \paragraph{}
        Este procesamiento de los datos fue realizado con el fin de poder analizar más intuitivamente la información presentada, 
        pudiendo así contrastar los diferentes escenarios conocidos contra la señal en estudio e identificar a cuál se asemeja.
        
        Una vez hecho esto, pudimos conjeturar que los valores presentes en la matriz de espectro representan la concentración de 
        energía en determinadas frecuencias de los valores que conforman los segmentos. Además, se pudo concluir que la señal propuesta 
        se asemeja a una función que varía linealmente en frecuencia.




    
\end{document}

